\section{Quantization of a single-mode field}

\subsection{Fields in a cavity}
Lets consider the following one-dimensional problem, where a cavity of length $L$ is oriented along the z-axis.

A linear polarized E-field is assumed, the medium is free space, perfect conducting walls and there is no free charges nor free current. The scheme is 
shown in figure \ref{fig:onedimensionalcavity}.
\begin{figure}[htbp]
    \centering
    \begin{tikzpicture}%y,z,x
        \draw[arrow,thick](0,0,0)--(12,0,0)node[below]{$z$};
        \draw[arrow,thick](0,0,0)--(0,2,0)node[right]{$z$};
        \draw[arrow,thick](0,0,0)--(0,0,2)node[below]{$x$};
        \draw[fill=NavyBlue!10,opacity=.5](0,1,1)--(0,-1,1)--(0,-1,-1)--(0,1,-1)--cycle;
        \draw[very thick,NavyBlue](0,0,0)plot[domain=0:10,samples=100]({\x},{1*sin(pi*\x/10 r)},0);
        \draw[very thick,OliveGreen](0,0,0)plot[domain=0:10,samples=100]({\x},{1*sin(pi*\x/5 r)},0);
        \draw[very thick,YellowOrange](0,0,0)plot[domain=0:10,samples=200]({\x},{1*sin(pi*\x r)},0);
        \draw[fill=NavyBlue!10,opacity=.5](10,1,1)--(10,-1,1)--(10,-1,-1)--(10,1,-1)--cycle;
        \draw[|-|,thick](0,1.5,0)--(10,1.5,0)node[midway,fill=white]{$L$};
        \draw(0,-.5,.5)node[]{$A$};
    \end{tikzpicture}
    \caption{One-dimensional cavity problem. Perfect conducting walls.}
    \label{fig:onedimensionalcavity}
\end{figure}

Our goal is to find the E- and B-field inside the cavity. Maxwell's equations in this case are:
\begin{align}
    \shortstack{Maxwell's equations\\with free sources}\qquad\left\{
    \begin{array}{l}
        \nabla\cdot\bE=0\\
        \nabla\cdot\bB=0\\
        \nabla\times\bE=-\partial_t\bB\\
        \nabla\times\bB=\frac{1}{c^2}\partial_t\bE
    \end{array}\right.
\end{align}
The E-field will be assumed to be $\bE(z,t)=\be E(z,t)$, where $\be$ is the polarization vector. Because fields depends only on z, $\nabla=\hz\partial_z$.
First Maxwell equation yields:
\begin{align*}
    \nabla\cdot\bE=\partial_z(\hz\cdot\bE)=\partial_z(\hz\cdot\be E)=0\Longrightarrow\be\cdot\hz=0.
\end{align*}
This implies that the polarization vector must be unitary in the transverse plane:
\begin{align*}
    \be=\cos\phi\hx+\sin\phi\hy.
\end{align*}
Third Maxwell's equation yields 
\begin{align*}
    \nabla\times\bE=(\hz\partial_z)\times(\be E)=(\hz\times\be)\partial_zE=-\partial_t\bB.
\end{align*}
Taking the curl of this equation, using Fourth Maxwell's equation and vector identities:
\begin{align*}
    \nabla\times(\nabla\times\bE)&=-\partial_t(\nabla\times\bB)\\
    \nabla(\cancelto{0}{\nabla\cdot\bE})-\nabla^2\bE&=-\frac{1}{c^2}\partial_t^2\bE\\
    -\be\partial_z^2E&=-\be\frac{1}{c^2}\partial_t^2E.
\end{align*}
From here, we have the E-field wave equation for this particular problem.
\begin{align}
    \partial_z^2E-\frac{1}{c^2}\partial_t^2E=0.
    \label{eq:waveequation_cavity}
\end{align}
Before solving this equation, we need the boundary condition set by the PEC condition. We need to $\hn\times\bE=0$ on the surface. Because 
the normal surface unit vector is $\hn=\pm\hz$, we have 
\begin{align*}
    \text{Boundary condition}\qquad\hn\times\bE=\hz\times(\be E)=0\Longrightarrow E(z=0,t)=E(z=L,t)=0.
\end{align*} 
In order to solve the PDE, we assum a product form $E(z,t)=Z(z)q(t)$. Then, by replacing in it in \eqref{eq:waveequation_cavity}:
\begin{align*}
    \partial_z^2[Z(z)q(t)]-\frac{1}{c^2}\partial_t^2[Z(z)q(t)]&=0\\
    Z''(z)q(t)-\frac{1}{c^2}Z(z)\ddot{q}(t)&=\biggr/[Z(z)q(t)]^{-1}\\
    \frac{Z''(z)}{Z(z)}&=\frac{1}{c^2}\frac{\ddot{q}(z)}{q(z)}.
\end{align*}
Left side depends only on $z$, while the right side only on $y$. The only way this can be true is if both are a constant, say, $-k^2$. Then,
\begin{align*}
    \text{Spatial and temporal differential equations}\qquad
    \left\{\begin{array}{l}
        \dfrac{Z''}{Z}=-k^2\longrightarrow Z''+k^2Z=0\\
        \dfrac{1}{c^2}\dfrac{\ddot{T}}{T}=-k^2\longrightarrow\ddot{q}+\omega^2q(t)=0
    \end{array}\right..
\end{align*} 
For the spatial ODE, we assume a solution of the form 
\begin{align*}
    Z(z)&=A\sin(kz)+B\cos(kz),\quad Z(0)=Z(L)=0.
\end{align*}
Setting the boundaries:
\begin{align*}
    Z(0)&=A\sin(0)+B\cos(0)=0\Longrightarrow B=0\\
    Z(L)&=A\sin(kL)=0\Longrightarrow k_m=\frac{m\pi}{L},\quad m\in\mathbb{N}.
\end{align*}
We left the temporal ODE unsolved.
Finally, putting all together yields the initial E-field:
\begin{align*}
    \bE_{\bk,\lambda}(z,t)=\be_{\lambda}\textcolor{BrickRed}{\sqrt{\frac{2\omega^2}{V\varepsilon_0}}}q_{\bk,\lambda}(t)\sin(kz),\quad k_m=\frac{m\pi}{L},\quad\omega=ck.
    \label{eq:Efieldincavity_generalform}
\end{align*}
We have includnig the subscript $\lambda$ and $\bk$ to consider multiple mode $\bk$ varied with $m$ and $\lambda$. Also, the coefficient in red is for better 
results in the future.
Using Faraday's law:
\begin{align*}
    \bB_{\bk,\lambda}(z,t)=(\bk\times\be_{\lambda})\frac{1}{kc^2}\sqrt{\frac{2\omega^2}{V\varepsilon_0}}\dot{q}_{\bk,\lambda}(t)\sin(kz).
\end{align*}
The term $\dot{q}(t)$ will play the role of a canonical momentum for a particle of unit mass, $p(t)=\dot{q}(t)$.
%%
\subsection{Single-mode Hamiltonian}
The classical field energy, or Hamiltonian $H$, of the single-mode field is given by 
\begin{align*}
    H_{\bk,\lambda}&=\frac{1}{2}\int\;dV\left[\varepsilon_0\bE_{\bk,\lambda}^2+\frac{1}{\mu_0}\bB_{\bk,\lambda}^2\right]\\
    &=\frac{1}{2}A\int_0^L\;dz\left[\frac{2\omega^2}{V\varepsilon_0}q_{\bk,\lambda}^2(t)\sin^2(kz)+\frac{1}{\mu_0}\frac{1}{k^2c^4}\frac{2\omega^2}{V\varepsilon_0}p_{\bk,\lambda}^2(t)\cos^2(kz)\right]\\
    &=\frac{A}{V}\int_0^L\;dz\left[\omega^2q_{\bk,\lambda}^2(t)\sin^2(kz)+p_{\bk,\lambda}^2(t)\cos^2(kz)\right]\\
    &=\frac{1}{L}\left[\omega^2q_{\bk,\lambda}^2(t)\frac{L}{2}+p_{\bk,\lambda}^2(t)\frac{L}{2}\right]\\
    H_{\bk,\lambda}&=\frac{1}{2}\left[\omega^2q_{\bk,\lambda}^2(t)+p_{\bk,\lambda}^2(t)\right].
\end{align*}
It is apparent that a single-mode field if formally equivalent to a harmonic quantim oscillator of unit mass, where the E- and B-fields play the 
roles of canonical position and momentum. 
\begin{align*}
    H_{\text{harmonic oscillator}}=\frac{1}{2}m\omega^2x^2+\frac{p^2}{2m}.
\end{align*}
To begin the quantization, we make $q,p$ operators $\hq,\hp$, which needs to satisfy the canonical commutation relations 
\begin{align}
    [\hq_{\bk,\lambda},\hp_{\bk,\lambda}]=i\hbar.
\end{align}
The EM fields with the operators are:
\begin{align}
    \shortstack{Quantized EM fields}\qquad\begin{array}{l}
        \bhE_{\bk,\lambda}(z,t)=\be_{\bk,\lambda}\sqrt{\dfrac{2\omega^2}{V\varepsilon_0}}\hq(t)\sin(kz)\\
        \bhB_{\bk,\lambda}(z,t)=(\bk\times\be_{\bk,\lambda})\dfrac{1}{kc^2}\sqrt{\dfrac{2\omega^2}{V\varepsilon_0}}\hp(t)\cos(kz)
    \end{array}.
\end{align}
The operators $\hq,\hp$ are Hermitian and therefore correspond to obserable quantites. It is convenient to introduce the ladder operators to 
simplify the expresison:
\begin{align*}
    \text{Ladder operator}\qquad
    \begin{array}{l}
        \ha_{\bk,\lambda}=\sqrt{\frac{1}{2\hbar\omega}}(\omega\hq_{\bk,\lambda}+i\hp_{\bk,\lambda})\\
        \ha_{\bk,\lambda}^\dagger=\sqrt{\frac{1}{2\hbar\omega}}(\omega\hq_{\bk,\lambda}-i\hp_{\bk,\lambda})
    \end{array}\longrightarrow
    \begin{array}{l}
        \hq_{\bk,\lambda}=\sqrt{\frac{\hbar}{2\omega}}(\ha_{\bk,\lambda}+\ha_{\bk,\lambda}^\dagger)\\
        \hp_{\bk,\lambda}=-i\sqrt{\frac{\hbar\omega}{2}}(\ha_{\bk,\lambda}-\ha_{\bk,\lambda}^\dagger)
    \end{array}.
\end{align*}
Using these defintion, the Hamiltonian yields:
\begin{align*}
    \hH_{\bk,\lambda}&=-\frac{1}{2}\frac{\hbar\omega}{2}(\ha_{\bk,\lambda}-\ha_{\bk,\lambda}^\dagger)(\ha_{\bk,\lambda}-\ha_{\bk,\lambda}^\dagger)+\frac{1}{2}\frac{\hbar\omega^2}{2\omega}(\ha_{\bk,\lambda}+\ha_{\bk,\lambda}^\dagger)(\ha_{\bk,\lambda}+\ha_{\bk,\lambda}^\dagger)\\
    &=\frac{\hbar\omega}{2}[\ha_{\bk,\lambda}^\dagger\ha_{\bk,\lambda}+\ha_{\bk,\lambda}\ha_{\bk,\lambda}^\dagger]=\hbar\omega\left[\ha_{\bk,\lambda}^\dagger\ha_{\bk,\lambda}+\frac{1}{2}\right]=\hbar\omega\left[n_{\bk,\lambda}+\frac{1}{2}\right].
\end{align*}
The time dependence of the ladder operator in Heisenberg equation is:
\begin{align*}
    \frac{d\ha_{\bk,\lambda}}{dt}=\frac{i}{\hbar}[\hH_{\bk,\lambda},\ha_{\bk,\lambda}]=\frac{i}{\hbar}\left[\hbar\omega\left(\ha_{\bk,\lambda}^\dagger
    \ha_{\bk,\lambda}+\frac{1}{2}\right),\ha_{\bk,\lambda}\right]=-i\omega\ha_{\bk,\lambda}\longrightarrow
    \begin{array}{l}
        \ha_{\bk,\lambda}(t)=\ha_{\bk,\lambda}(0)e^{-i\omega t}\\
        \ha_{\bk,\lambda}^\dagger(t)=\ha_{\bk,\lambda}^\dagger(0)e^{i\omega t}
    \end{array}.
\end{align*}
For example, we have the following applications:
\begin{align*}
    \ha_{\bk,\lambda}\ket{0}=\ket{1}_{\bk,\lambda},\quad\ha_{\bk,\lambda}\ket{n}_{\bk,\lambda}=\sqrt{n}\ket{n-1}_{\bk,\lambda},
    \quad\ha^\dagger_{\bk,\lambda}\ket{n}_{\bk',\lambda'}=\ket{n}_{\bk,\lambda}\ket{1}_{\bk',\lambda'},\quad\ha_{\bk,\lambda}\ket{n}_{\bk',\lambda'}\ket{0}_{\bk,\omega}=0.
\end{align*}

The operator product $\ha^\dagger\ha$ has an important significance and is called the number operator $\hn$, whose eigenequation is 
\begin{align*}
    \hn\ket{n}=n\ket{n}.
\end{align*}
The state $\ket{n}$ is the energy eigenstte of the single mode field with energy eiganvlaue $E_n$:
\begin{align*}
    \hH\ket{n}=\hbar\omega(\ha^\dagger\ha+\frac{1}{2})\ket{n}=E_n\ket{n}
\end{align*}
The relation between the number state and the raising operator $\ha^\dagger$ is:
\begin{align*}
    \ket{n}=\frac{(\ha^\dagger)^n}{\sqrt{n}}\ket{0}.
\end{align*}

%%
\subsection{Quantizing the multimode field}
We now consider all the modes in the optical cavity:
\begin{align}
    \bE(z,t)&=\sum_{m,\lambda}\he_\lambda\sqrt{\frac{2\omega_m^2}{\varepsilon_0V}}\hq_{m,\lambda}(t)\sin(k_mz)\\
    \bB(z,t)&=\sum_{m,\lambda}(\hz\times\he_\lambda)\frac{1}{k_mc^2}\sqrt{\frac{2\omega_m^2}{\varepsilon_0V}}\dot{\hq}_{m,\lambda}(t)\cos(k_mz).
\end{align}
where $kL=m\pi$ and $\he_\lambda=\{\hx,\hy\}$. The process is anaglogous; we compute the Hamiltonian:
\begin{align*}
    H&=\int\;dV\left[\frac{1}{2}\varepsilon_0\bE^2(z,t)+\frac{1}{2\mu_0}\bB(z,t)\right]\\
    &=\int\;dV\left[\frac{1}{2}\varepsilon_0\left\{\sum_{m,\alpha}\he_\alpha\sqrt{\frac{2\omega_m^2}{\varepsilon_0V}}q_{m,\alpha}(t)\sin(k_mz)\right\}^2
    +\frac{1}{2\mu_0}\varepsilon_0\left\{\sum_{m,\alpha}(\hz\times\he_\alpha)\frac{\mu_0}{\varepsilon_0}{k_m}\sqrt{\frac{2\omega_m^2}{\varepsilon_0V}}\dot{q}_{m,\alpha}(t)\cos(k_mz)\right\}^2\right]\\
    &\vdots\\
    H&=\frac{1}{2}\sum_{m,\lambda}(\omega_m^2q_{m,\lambda}^2+\dot{q}_{m,\lambda}^2).
\end{align*}
That is,
\begin{emphasizer}
    Each mode of the EM field is an independent harmonic oscillator.
\end{emphasizer}
We can quantize the multimode EM field in a similar way as the sige mode field to obtain the Hamiltonian:
\begin{align*}
    \text{Hamiltoninan multimode EM field}\qquad\hH=\sum_{m,\lambda}\hbar\omega_{m}\left[\ha^\dagger_{m,\lambda}a_{m,\lambda}+\frac{1}{2}\right]=\sum_{m,\lambda}\hbar\omega_m\left[\hn_{m,\lambda}+\frac{1}{2}\right],
\end{align*}
where $\hn_{m,\lambda}$ is the number of excitations in mode $m,\lambda$.

%%
\subsection{Quantizing EM field in free space}
We define the scalar and vector potential 
\begin{align*}
    \bB=\nabla\times\bA\quad\text{and}\quad\bE=-\partial_t\bA-\nabla\phi.
\end{align*}
Using these automatically satisfy the Gauss magnetic equation and Ampere law. In QO, we assume \bfemph{Coulomb gauge}: $\nabla\cdot\bA=0$.
With this, the E-field is 
\begin{align*}
    \bE=-\partial_t\bA-\nabla\phi=\bE^\perp+\bE^\parallel,\quad\text{where}\begin{cases}
        \nabla\cdot\bE^\perp=0,&\text{$\bA$ corresponds to radiation (all transverse)}\\
        \bE''=-\nabla\phi,&\text{$\bA$ corresponds to the field of sources (all longitudinal)}
    \end{cases}.
\end{align*}
In Coulomb gauge, in absence of charge $\phi=0$ such that the vector potential-wave equation is 
\begin{align*}
    \nabla^2\bA-\frac{1}{2}\partial_t^2\bA=0.
\end{align*}
It can be solved with the following plane wave 
\begin{align*}
    \bA(\br,t)=\sum_{\bk,\lambda}\he_{\bk,\lambda}\left[A_{\bk,\lambda}e^{i(\bk\cdot\br-\omega t)}+A^*_{\bk,\lambda}e^{-i(\bk\cdot\br-\omega t)}\right],\quad\nabla\cdot\bA=0\Longrightarrow\bk\cdot\he_{\bk,\lambda}=0.
\end{align*}
The vector potential in free space is expressed as a sueprposition of plane waves such that linear dispersion $\omega=|\bk|c$ is satisfied with 
$\he_{\bk,\lambda}\cdot\he_{\bk,\lambda'}=\delta_{\lambda\lambda'}$.

The quantization volume we quantize the field is considered to have finite volume of $V$ and we impose periodic boundary such that each side is equal.

The E- and B-field are then:
\begin{align*}
    \bE(\br,t)&=-\partial_t\bA(\br,t)=i\sum_{\bk,\lambda}\he_{\bk,\lambda}\omega\left[A_{\bk,\lambda}e^{i(\bk\cdot\br-\omega t)}+A^*_{\bk,\lambda}e^{-i(\bk\cdot\br-\omega t)}\right],\\
    \bB(\br,t)&=\nabla\times\bA(\br,t)=i\sum_{\bk,\lambda}\he_{\bk,\lambda}\left[A_{\bk,\lambda}e^{i(\bk\cdot\br-\omega t)}+A^*_{\bk,\lambda}e^{-i(\bk\cdot\br-\omega t)}\right].
\end{align*}
The energy (Hamiltonian) associated with the fields is:
\begin{align*}
    H&=-\int\;dV\left[\frac{1}{2}\varepsilon_0\bE^2(\br,t)+\frac{1}{2\mu_0}\bB(\br,t)\right]\\
    &\vdots\\
    H&=2\varepsilon_0V\sum_{\bk,\lambda}\omega^2|A_{\bk,\lambda}|^2.
\end{align*}
We have used the following relation:
\begin{align*}
    \int\;dVe^{\pm i(\bk-\bk')\cdot\br}=\delta_{\bk\bk'}V.
\end{align*}
Defining 
\begin{align*}
    A_{\bk,\lambda}=\frac{1}{2\omega\sqrt{\varepsilon_0V}}(\omega q_{\bk,\lambda}+ip_{\bk,\lambda}),\quad
    A^*_{\bk,\lambda}=\frac{1}{2\omega\sqrt{\varepsilon_0V}}(\omega q_{\bk,\lambda}-ip_{\bk,\lambda}).
\end{align*}
The Hamiltonian is:
\begin{align*}
    H=\frac{1}{2}\sum_{\bk,\lambda}(\omega^2q^2_{\bk,\lambda}+p^2_{\bk,\lambda})\xrightarrow{\text{quantizing}}\hH=\sum_{\bk,\lambda}\hbar\omega\left(\ha^\dagger_{\bk,\lambda}
    \ha_{\bk,\lambda}+\frac{1}{2}\right),
\end{align*}
where we have defined the ladder operators:
\begin{align}
    \ha=\frac{1}{\sqrt{2\hbar\omega}}(\omega\hq+i\hp),\quad\ha^\dagger=\frac{1}{\sqrt{2\hbar\omega}}(\omega\hq-i\hp).
\end{align}
So, the coefficients are expressed in term of the ladder:
\begin{align*}
    A_{\bk,\lambda}=\sqrt{\frac{\hbar}{2\omega\varepsilon_0V}}\ha_{\bk,\lambda},\quad A^*_{\bk,\lambda}=\sqrt{\frac{\hbar}{2\omega\varepsilon_0V}}\ha^\dagger_{\bk,\lambda}.
\end{align*}
Thus the fields are:
\begin{align}
    \begin{array}{l}
    \bA(\br,t)=\sum_{\bk,\lambda}\sqrt{\frac{\hbar}{2\omega\varepsilon_0V}}\he_{\bk,\lambda}\left[\ha_{\bk,\lambda}e^{i(\bk\cdot\br-\omega t)}+\ha^\dagger_{\bk,\lambda}e^{-i(\bk\cdot\br-\omega t)}\right],\\
    \bE(\br,t)=i\sum_{\bk,\lambda}\sqrt{\frac{\hbar\omega}{2\varepsilon_0V}}\he_{\bk,\lambda}\left[\ha_{\bk,\lambda}e^{i(\bk\cdot\br-\omega t)}+\ha^\dagger_{\bk,\lambda}e^{-i(\bk\cdot\br-\omega t)}\right],\\
    \bB(\br,t)=\frac{i}{c}\sum_{\bk,\lambda}\sqrt{\frac{\hbar\omega}{2\varepsilon_0V}}(\bk\times\he_{\bk,\lambda})\left[\ha_{\bk,\lambda}e^{i(\bk\cdot\br-\omega t)}+\ha^\dagger_{\bk,\lambda}e^{-i(\bk\cdot\br-\omega t)}\right].
    \end{array}\label{eq:emfieldquantization}
\end{align}


\subsection{Electric field expectation and uncertainty}
Lets conside one mode of the E-field in a number state: $\ket{n}_{\bk,\lambda}$.
Lets add the time dependence to the ladder operators:
\begin{align*}
    \ha_{\bk,\lambda}e^{-i\omega t}\to\ha_{\bk,\lambda}(t),\quad\ha^\dagger_{\bk,\lambda}e^{i\omega t}\to\ha^\dagger_{\bk,\lambda}(t).
\end{align*}
The E-field is:
\begin{align*}
    \hbE(\br,t)=i\sum_{\bk,\lambda}e_{\bk,\lambda}\sqrt{\frac{\hbar\omega}{2\varepsilon_0V}}\left[\ha_{\bk,\lambda}(t)e^{i\bk\cdot\br}-\ha^\dagger_{\bk,\lambda}(t)
    e^{-i\bk\cdot\br}\right].
\end{align*}
Expectation value of the E-field is:
\begin{align*}
    _{\bk_0,\lambda_0}\braket{n|\hbE(\br,t)|n}_{\bk_0,\lambda_0}&=i_{\bk,\lambda}\braket{n|\he_{\bk,\lambda}\sqrt{\frac{\hbar\omega}{2\varepsilon_0V}}\left[\ha_{\bk,\lambda}(t)e^{i\bk\cdot\br}-\ha^\dagger_{\bk,\lambda}(t)e^{-i\bk\cdot\br}\right]|n}_{\bk_0,\lambda_0}\\
    &=i_{\bk_0,\lambda_0}\bra{n}\he_{\bk_0,\lambda_0}\sqrt{\frac{\hbar\omega_0}{2\varepsilon_0V}}\left[\sqrt{n}\ket{n-1}_{\bk_0,\lambda_0}e^{i\bk_0\cdot\br}-\sqrt{n+1}\ket{n+1}_{\bk_0,\lambda_0}e^{-i\bk_0\cdot\br}\right]\\
    &=0.
\end{align*}
Thus, the average E-field is zero.

Lets consider the uncertainty of the E-field:
\begin{align*}
    \braket{\hbE}&=_{\bk_0,\lambda_0}\braket{n|\sum_{\bk,\lambda}\sum_{\bk',\lambda'}\sqrt{\frac{\hbar\omega}{2\varepsilon_0V}}\sqrt{\frac{\hbar\omega'}{2\varepsilon_0V}}\he_{\bk,\lambda}\cdot\he_{\bk',\lambda'}
    \left[\ha_{\bk,\lambda}e^{i\bk\cdot\br}-\ha^\dagger_{\bk,\lambda}e^{-i\bk\cdot\br}\right]\left[\ha_{\bk',\lambda'}e^{i\bk'\cdot\br}-\ha^\dagger_{\bk',\lambda'}e^{-i\bk'\cdot\br}\right]|n}_{\bk_0,\lambda_0}\\
    &=\sum_{\bk,\lambda}\sum_{\bk',\lambda'}\sqrt{\frac{\hbar\omega}{2\varepsilon_0V}}\sqrt{\frac{\hbar\omega'}{2\varepsilon_V}}_{\bk,\lambda}\braket{1|1}_{\bk',\lambda'}e^{i(\bk-\bk')\cdot\br}+
    \frac{\hbar\omega_0}{2\varepsilon_0V}[_{\bk_0,\lambda_0}\bra{n+1}\sqrt{n+1}+_{\bk_0,\lambda_0}\bra{n-1}\sqrt{n}][\sqrt{n+1}\ket{n+1}_{\bk_0,\lambda_0}+\sqrt{n}\ket{n-1}_{\bk_0,\lambda_0}]\\
    &=\sum_{\{\bk,\lambda\}\neq\{\bk_0,\lambda_0\}}\frac{\hbar\omega}{2\varepsilon_0V}+\frac{\hbar\omega_0}{2\varepsilon_0V}(2n+1).
\end{align*}
So, the uncertainty is:
\begin{align*}
    \text{E-field uncertainty}\qquad\highlight{\braket{\Delta E^2}=\underbrace{\sum_{\{\bk,\lambda\}\neq\{\bk_0,\lambda_0\}}\frac{\hbar\omega}{2\varepsilon_0V}}_{\text{All modes in vaccum state}}+\underbrace{\frac{\hbar\omega_0}{2\varepsilon_0V}(2n+1)}_{\text{Mode ($\bk_0,\lambda_0$)}}}.
\end{align*}

