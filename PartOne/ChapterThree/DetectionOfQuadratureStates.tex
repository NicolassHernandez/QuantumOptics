\section{Detection of quadrature states}
Modes at the beam splitter output are:
\begin{align*}
    \hA_1=\frac{1}{\sqrt{2}}(\ha_1+\ha_2),\quad\hA_2=\frac{1}{\sqrt{2}}(\ha_1-\ha_2).
\end{align*}
Detectors $D_1,D_2$ measure the intensity at each output:
\begin{align*}
    \braket{\hat{I}_{1,1}}=\braket{\hA^\dagger_{1,2}\hA_{1,2}}.
\end{align*}
\begin{figure}[htbp]
    \centering
    \includegraphics[width=.5\columnwidth]{PartOne/ChapterThree/figures/homodynedetection.png}
    \caption{Balanced homodyne detection.}
\end{figure}

In output 1, we have 
\begin{align*}
    I_1=\braket{\psi_1\alpha_2|\hA_{1}\hA_1|\alpha_2\Psi_1}&=\braket{\psi_1\alpha_2|\frac{\ha^\dagger_1+\ha^\dagger_2}{\sqrt{2}}\frac{\ha_1+\ha_2}{\sqrt{2}}|\alpha_2\Psi_1}=
    \frac{1}{2}\braket{\Psi_1\alpha_2|(\ha^\dagger_1+\alpha^*)(\ha_1+\alpha)|\alpha_2\Psi_1}\\
    &=\frac{1}{2}\braket{\Psi_1|(\ha^\dagger\ha_1+\ha_1\alpha^*+\ha^\dagger_1\alpha+|\alpha|^2)|\Psi_1}
\end{align*}
Lets consider that input 2 contains a strong coherent state field that we refer to as the \emph{local oscillator}. 
Thus the total state over which the average is being taken is 
\begin{align}
    \underbrace{\ket{\Psi_1}}_{\text{Input 1}}\underbrace{\ket{\alpha_2}}_{\text{Coherent state input 2}}.
\end{align}
The averaging if about these two states.

We see that the irradiance is just the number operator counting photons at each output.
For instance,
\begin{align*}
    I_1-I_2&=\braket{\Psi_1|\ha_1\alpha_2^*+\ha_1^\dagger\alpha_2|\Psi_1}=\begin{cases}
        2\alpha_2\braket{\Psi_1|\dfrac{\ha_1+\ha_1^\dagger}{2}|\Psi_1}=2\alpha_2\braket{\Psi_1|\hX_1|\Psi_1},&\alpha_2=\alpha_2^*\\
        -2i\alpha_2\braket{\Psi_1|\dfrac{\ha_1-\ha_1^\dagger}{2i}|\Psi_1}=-2i\alpha_2\braket{\Psi_1|\hX_2|\Psi_1},&\alpha_2=-\alpha_2^*
    \end{cases}
\end{align*}
In general by changing the phase of input 2 we can measure a general quadrature. Noise in intensity corresponds to noise in specific quadratures.
If a quadrature measurement has uncertainty $\Delta\hX^2<1/4$.


