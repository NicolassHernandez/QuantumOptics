\section{Beam splitter}

Let us consider the action of a beam splitter (BS) on two input modes of same frequency and polarization.

Clasically, the output E-field can be defined in terms of the input E-fields:
\begin{align*}
    \begin{bmatrix}
        E_1'\\E'_2
    \end{bmatrix}=\begin{bmatrix}
        t'&r\\r'&t
    \end{bmatrix}\begin{bmatrix}
        E_1\\E_2
    \end{bmatrix}.
\end{align*}
$\{r,t\}$ and $\{r',t'\}$ are reflection/transmission coefficient associated with the two interfaces of the BS. 
From energy conservation, we must have 
\begin{align*}
    \text{Energy conservation in the BS}\qquad\highlight{|E_1|^2+|E_2|^2=|E_1'|^2+|E_2'|^2}.
\end{align*}
If we assume $E_2=0$, then 
\begin{align}
    |E_1|^2=|t'E_1|^2+|r'E_1|^2\longrightarrow|r'|^2+|t'|^2=1.\label{eq:E20}
\end{align} 
Similarly, setting $E_1=0$ yields 
\begin{align}
    |E_2|^2=|rE_2|^2+|tE_2|^2\longrightarrow|r|^2+|t|^2=1.\label{eq:E10}
\end{align}
In general,
\begin{align*}
    |E_1|^2+|E_2|^2&=|E_1'|^2+|E_2'|^2\\
    &=(t'E_1+rE_2)(t^{'*}E_1^*+r^*E_2^*)+(r'E_1+tE_2)(r^{'*}E_1^*+t^*E_2^*)\\
    &=|t'|^2|E_1|^2+|r|^2|E_2|^2+|r'|^2|E_1|^2+|t|^2|E_2|^2+rt^{'*}E_1^*E_2+t'r^*E_1E_2^*+r't^*E_1E_2^*+r^{'*}tE_1^*E_2\\
    |E_1|^2+|E_2|^2&=|E_1|^2(|r'|^2+|t'|^2)+|E_2|^2(|r|^2+|t|^2)+E_1^*E_2(rt^{'*}+r^{'*}t)+E_1E_2^*(r't^*+r^*t')\\
    0&=E_1^*E_2(rt^{'*}+r^{'*}t)+c.c.
\end{align*}
Thus, 
\begin{align}
    rt^{'*}+r^{'*}t=0.\label{eq:E12}
\end{align}
For a $50:50$ BS which as $r'=1/\sqrt{2}$, $t'=1/\sqrt{2}$, and $r=1/\sqrt{2}$, \eqref{eq:E20}, \eqref{eq:E10} and \eqref{eq:E12} implies 
that $t=-1/\sqrt{2}$. Similarly, the quantum description of the BS is 
\begin{align}
    \begin{bmatrix}
        \ha_1'\\\ha_2'
    \end{bmatrix}=\frac{1}{\sqrt{2}}\begin{bmatrix}
        1&1\\1&-1
    \end{bmatrix}\begin{bmatrix}
        \ha_1\\\ha_2
    \end{bmatrix}.
\end{align}
The ladder operators are equal to the E-fields. In addition, the output modes are written as a linear combination of the input modes:
\begin{align*}
    \begin{array}{l}
        \ha_1'=\frac{1}{\sqrt{2}}(\ha_1+\ha_2)\\
        \ha_2'=\frac{1}{\sqrt{2}}(\ha_1-\ha_2)
    \end{array}
    \longrightarrow
    \begin{array}{l}
        \ha_1=\frac{1}{\sqrt{2}}(\ha_1'+\ha_2')\\
        \ha_2=\frac{1}{\sqrt{2}}(\ha_1'-\ha_2')
    \end{array}
\end{align*}

\begin{figure}[htbp]
    \centering
    \begin{subfigure}{.45\columnwidth}
        \centering
        \includegraphics[width=\columnwidth]{PartOne/ChapterTwo/figures/bs_classical.png}
        \caption{Classical description}
    \end{subfigure}
    \hfill
    \begin{subfigure}{.45\columnwidth}
        \centering
        \includegraphics[width=\columnwidth]{PartOne/ChapterTwo/figures/bs_quantum.png}
        \caption{Quantum description}
    \end{subfigure}
    \caption{Description of a beamsplitter (BS).}
\end{figure}


\begin{example}{Single photon incident in one input}
    The input is:
    \begin{align*}
        \text{Input}=\ket{1}_1\ket{0}_2=(\ha^\dagger_1\ket{0}_1)\ket{0}_2=\ket{1}_1\ket{0}_2.
    \end{align*}
    Note that 
    \begin{align*}
        \ha_1=\frac{1}{\sqrt{2}}(\ha_1'+\ha_2')\quad\text{and}\quad\ha_1^\dagger=\frac{1}{\sqrt{2}}(\ha_1^{'\dagger}+\ha_2^{'\dagger}).
    \end{align*}
    The output is then 
    \begin{align*}
        \text{Output}=\frac{1}{\sqrt{2}}(\ha_1^{'\dagger}+\ha_2^{'\dagger})\ket{0}_{1'}\ket{0}_{2'}=\frac{1}{\sqrt{2}}[\ket{1}_{1'}\ket{0}_{2'}+\ket{0}_{1'}\ket{1}_{2'}].
    \end{align*}
    Thus, there is a $50:50$ probability of detecting the photon in one of the ouput ports.
\end{example}
\begin{example}{Both input with photons}
    The input is:
    \begin{align*}
        \text{Input}=\ket{1}_1\ket{1}_2=\ha_1^\dagger\ket{0}_1\ha_2^\dagger\ket{0}_2.
    \end{align*}
    The output is:
    \begin{align*}
        \text{Output}&=\frac{1}{2}[\ha_1^{'\dagger}+\ha_2^{'\dagger}][\ha_1^{'\dagger}-\ha_2^{'\dagger}]\ket{0}_{1'}\ket{0}_{2'}\\
        &=\frac{1}{2}[\ha_1^{'\dagger}\ha_1^{'\dagger}-\ha_2^{'\dagger}\ha_2^{'\dagger}+\cancel{\ha_1^{'\dagger}\ha_2^{'\dagger}}-\cancel{\ha_2^{'\dagger}\ha_1^{'\dagger}}]\ket{0}_{1'}\ket{0}_{2'}\\
        &=\frac{1}{2}[\sqrt{2}\ket{2}_{1'}\ket{0}_{2'}-\sqrt{2}\ket{0}_{1'}\ket{2}_{2'}]\\
        \text{Output}&=\frac{1}{\sqrt{2}}[\ket{2}_{1'}\ket{0}_{2'}-\ket{0}_{1'}\ket{2}_{2'}].
    \end{align*}
    Thus, when two photons are simultaneously incident on each input, both either go into output 1 or output 2 but never in both a the same time.
\end{example}



\begin{example}{Mach-Zehnder}
    \begin{figure}[htbp]
        \centering
        \includegraphics[width=.4\columnwidth]{PartOne/ChapterTwo/figures/example_machzenhder.png}
    \end{figure}
    \begin{enumerate}[itemsep=0pt,topsep=0pt,parsep=0pt,label=\alph*)]
        \item Express the output modes $\{\hb_1,\hb_2\}$ in terms of the input modes $\{\ha_1,\ha_2\}$ assuming the first BS is $50:50$.
        \item What is the expected number of photons $\braket{\hb_i^\dagger\hb_i}$ in each mode?
        \item If $\hb_1\to\hb_1e^{i\phi}$, determine the output mode operators are the detector positions $\hc_1=\frac{1}{\sqrt{2}}(\hb_1e^{i\phi}+\hb_2)$ 
        and $\hb_2=\frac{1}{\sqrt{2}}(\hb_1e^{i\phi}-\hb_2)$ in terms of the input mode operators $\{\ha_1,\ha_2\}$.
        \item What is the number of photons detected at detector $1$ as a function of $\phi$?
    \end{enumerate}
    \subsubsection{Solution}
\begin{enumerate}[itemsep=0pt,topsep=0pt,parsep=0pt,label=(\alph*)]
  \item For the $50:50$ beamsplitter 1, we have:
  \begin{align*}
    \begin{bmatrix}
      \hb_1\\\hb_2
    \end{bmatrix}=\frac{1}{\sqrt{2}}\begin{bmatrix}
      1&1\\1&-1
    \end{bmatrix}\begin{bmatrix}
      \ha_1\\\ha_2
    \end{bmatrix},\quad\text{with}\quad\begin{bmatrix}
      \ha_1\\\ha_2
    \end{bmatrix}=\frac{1}{\sqrt{2}}\begin{bmatrix}
      1&1\\1&-1
    \end{bmatrix}\begin{bmatrix}
      \hb_1\\\hb_2
    \end{bmatrix}.
  \end{align*}
  \item From the last part, we know that 
  \begin{align*}
    \ha_1=\frac{1}{\sqrt{2}}(\hb_1+\hb_2)\xrightarrow{h.c.}\ha_1^\dagger=\frac{1}{\sqrt{2}}(\hb_1^\dagger+\hb_2^\dagger),\quad\text{and}\quad
    \ha_2=\frac{1}{\sqrt{2}}(\hb_1-\hb_2)\xrightarrow{h.c.}\ha_2^\dagger=\frac{1}{\sqrt{2}}(\hb_1^\dagger-\hb_2^\dagger).
  \end{align*}
  The input is then:
  \begin{align*}
    \text{Input}=\ket{1}_1\ket{0}_2=\ha^\dagger_1\ket{0}_1\ket{0}_2.
  \end{align*}
  Output is:
  \begin{align*}
    \text{Output}=\frac{1}{\sqrt{2}}(\hb_1^\dagger+\hb_2^\dagger)\ket{0}_1\ket{0}_2=\frac{1}{\sqrt{2}}[\ket{1}_{1'}\ket{0}_{2'}+\ket{0}_{1'}\ket{1}_{2'}].
  \end{align*}
  We expect then to have the same probability of getting one photon at each output. 

  The expected value at output 1 is:
  \begin{align*}
    _2\bra{0}_1\bra{1}\hb_1^\dagger\hb_1\ket{1}_1\ket{0}_2&=\frac{1}{2}\braket{0,1|(\ha_1^\dagger+\ha_2^\dagger)(\ha_1+\ha_2)|1,0}\\
    &=\frac{1}{2}\bra{0,1}(\ha_1^\dagger\ha_1+\ha_1^\dagger\ha_2+\ha_2^\dagger\ha_1+\ha_2^\dagger\ha_2)\ket{1,0}\\
    _2\bra{0}_1\bra{1}\hb_1^\dagger\hb_1\ket{1}_1\ket{0}_2&=\frac{1}{2}.
  \end{align*}
  Likewise, the expected value at output 2 is:
  \begin{align*}
    _2\bra{0}_1\bra{1}\hb_2^\dagger\hb_2\ket{1}_1\ket{0}_2&=\frac{1}{2}\braket{0,1|(\ha_1^\dagger-\ha_2^\dagger)(\ha_1-\ha_2)|1,0}\\
    &=\frac{1}{2}\bra{0,1}(\ha_1^\dagger\ha_1-\ha_1^\dagger\ha_2-\ha_2^\dagger\ha_1+\ha_2^\dagger\ha_2)\ket{1,0}\\
    _2\bra{0}_1\bra{1}\hb_2^\dagger\hb_2\ket{1}_1\ket{0}_2&=\frac{1}{2}.    
  \end{align*}
  \item We replace the first output as the input of the second output:
  \begin{align*}
    \hc_1&=\frac{1}{\sqrt{2}}[\hb_1e^{i\phi}+\hb_2]=\frac{1}{2}\left[(\ha_1+\ha_2)e^{i\phi}+(\ha_1-\ha_2)\right]=\frac{1}{2}[\ha_1(1+e^{i\phi})+\ha_2(e^{i\phi}-1)],\\
    \hc_2&=\frac{1}{\sqrt{2}}[\hb_1e^{i\phi}-\hb_2]=\frac{1}{2}\left[(\ha_1+\ha_2)e^{i\phi}-(\ha_1-\ha_2)\right]=\frac{1}{2}[\ha_1(e^{i\phi}-1)+\ha_2(1+e^{i\phi})].
  \end{align*}
  \item The expected number of photons at detector 1 is:
  \begin{align*}
    \braket{0,1|\hc_1^\dagger\hc_1|1,0}&=\frac{1}{4}\left\{[\ha_1^\dagger(1+e^{-i\phi})+\ha_2^\dagger(e^{-i\phi}-1)][\ha_1(1+e^{i\phi})+\ha_2(e^{i\phi}-1)]\right\}\\
    &=\frac{1}{4}\bra{0,1}\left\{\ha_1^\dagger\ha_1(\cdots)+\ha_1^\dagger\ha_2(\cdots)+\ha_2^\dagger\ha_1(\cdots)+\ha_2^\dagger\ha_2(\cdots)\right\}\ket{1,0}\\
    &=\frac{1}{4}(1+e^{-i\phi})(1+e^{i\phi})\\
    \braket{0,1|\hc_1^\dagger\hc_1|1,0}&=\frac{1}{2}[1+\cos\phi]=\cos^2\frac{\phi}{2}.
  \end{align*}
  By conservation of energy (or probability), the expected number of photon at detector 2 is:
  \begin{align*}
    \braket{0,1|\hc_2^\dagger\hc_2|1,0}&=\sin^2\frac{\phi}{2}.
  \end{align*}
\end{enumerate}
\end{example}