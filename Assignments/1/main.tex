\documentclass[letterpaper,11pt,twoside]{article}
\usepackage{graphicx} % Required for inserting images
\usepackage[table,xcdraw,dvipsnames]{xcolor}
\usepackage{amsmath,amsfonts,amssymb,amsthm}
\usepackage{listings}
\usepackage{lipsum}
\usepackage{hyperref}
\usepackage{enumitem}

\usepackage{tikz}
\usepackage[siunitx, RPvoltages]{circuitikz}
\usetikzlibrary{3d}
\usepackage{comment}
\usepackage{caption,subcaption}
\usepackage{pgfplots}
\pgfplotsset{compat=newest} % or a newer version if available
\usepgfplotslibrary{groupplots}
\usetikzlibrary{pgfplots.groupplots}
\usetikzlibrary{shapes.geometric, arrows}
\tikzstyle{arrow} = [->,>=stealth,shorten >=2pt]
\newcommand{\re}[1]{\text{Re}\left(#1\right)}
\newcommand{\im}[1]{\text{Im}\left(#1\right)}
\newcommand{\bA}{\bm{A}}
\newcommand{\bB}{\bm{B}}
\newcommand{\bC}{\bm{C}}
\newcommand{\bE}{\bm{E}}
\newcommand{\bH}{\bm{H}}
\newcommand{\bD}{\bm{D}}
\newcommand{\bO}{\bm{0}}
\newcommand{\br}{\bm{r}}
\newcommand{\bR}{\bm{R}}
\newcommand{\hA}{\hat{A}}
\newcommand{\hB}{\hat{B}}
\newcommand{\hC}{\hat{C}}
\newcommand{\ha}{\hat{a}}
\newcommand{\hn}{\bm{\hat{n}}}
\newcommand{\hr}{\bm{\hat{r}}}
\newcommand{\hx}{\bm{\hat{x}}}
\newcommand{\hy}{\bm{\hat{y}}}
\newcommand{\hz}{\bm{\hat{z}}}
\newcommand{\laplace}[1]{\mathscr{L}\left[#1\right]}
\newcommand{\ilaplace}[1]{\mathscr{L}^{-1}\left[#1\right]}
\newcommand{\fourier}[1]{\mathscr{F}\left[#1\right]}
\newcommand{\ifourier}[1]{\mathscr{F}^{-1}\left[#1\right]}
\newcommand{\ket}[1]{\left|#1\right\rangle}
\newcommand{\bra}[1]{\left\langle#1\right|}
\newcommand{\braket}[1]{\langle#1\rangle}
\newcommand{\D}{\mathcal{D}}
\usepackage{cancel}
\usepackage{bm}
\usepackage{fancyhdr}
\usepackage[utf8x]{inputenc}
\usepackage[T1]{fontenc}
\usepackage[margin=0.8in,top=1in,bottom=1in]{geometry}
%%%%%
\begin{filecontents*}{refs.bib}
@book{bornwolf,
  author    = {Born, M. and Wolf, E.},
  title     = {Principles of Optics},
  publisher = {Pergamon Press},
  edition   = {7},
  year      = {1999}
}
@book{hecht,
  author    = {Hecht, E.},
  title     = {Optics},
  publisher = {Addison-Wesley},
  edition   = {5},
  year      = {2016}
}
\end{filecontents*}
%
\newcommand{\institution}{University of Arizona}
\newcommand{\autor}{Nicolás Hernández Alegría}
\newcommand{\course}{OPTI 544 Quantum Optics}
\newcommand{\assignment}{Assignment 1}
%
\title{\textbf{\assignment}\\\course\\{\Large\institution}}
\author{\autor}
\date{\today\\Total time: 12 hours}
%
\renewcommand{\sectionmark}[1]{\markright{#1}}
\fancypagestyle{mainstyle}{
    \fancyhf{} % Clear all header and footer fields
    \fancyfoot[C]{\thepage}
    \fancyhead[LE,RO]{\course} % Section name on odd pages
    \fancyhead[LO,RE]{\assignment}
    % Optional: Thin rules
    \renewcommand{\headrulewidth}{0pt} % Header rule
    \renewcommand{\footrulewidth}{0pt} % No footer rule
}
%
\begin{document}

\pagestyle{mainstyle}
\maketitle
%%
\section*{Exercise 1}
\begin{enumerate}[itemsep=0pt,topsep=0pt]
  \item Ill use the definition of the E- and B-fields in terms of the vector potential.
  \begin{align*}
    \bE=-\partial_t\bA,\quad\text{and}\quad\bB=\nabla\times\bA.
  \end{align*} 
  Taking the curl of the B-field allows to replace the vector potential and have a triple product,
  which can be reexpressed using vector identities. Because $\nabla\times\bB$ is Ampere's law,
  it also depends on the E-field, where the above definition can be replaced.
  \begin{align*}
    \nabla\times\bB&=\frac{1}{c^2}\partial_t\bE\\
    \nabla\times\nabla\times\bA&=\frac{1}{c^2}\partial_t(-\partial_t\bA)\\
    \nabla(\cancelto{0}{\nabla\cdot\bA})-\nabla^2\bA&=-\frac{1}{c^2}\partial_t^2\bA\\
    \nabla^2\bA(\br,t)-\frac{1}{c^2}\partial_t^2\bA(\br,t)&=0.
  \end{align*}
  \item fsfaf
  \item asgsag
  \item asgsa
  \item sagasg
\end{enumerate}
%%
\section*{Exercise 2}

%%

\section*{Exercise 3}
\begin{enumerate}[itemsep=0pt,topsep=0pt,label=\alph*)]
  \item We can simplify the problem using the BCH formula
  \begin{align*}
    e^{\hA}\hB e^{-\hA}=\hB+[\hA,\hB]+\frac{1}{2!}[\hA,[\hA,\hB]]+\cdots
  \end{align*}
  In this case, $\hA=\alpha^*\ha-\alpha\ha^\dagger$ and $\hB=\ha$. The first two commutators are:
  \begin{align*}
    [\hA,\hB]&=[(\alpha^*\ha-\alpha\ha^\dagger),\ha]=\alpha^*[\ha,\ha]-\alpha[\ha^\dagger,\ha]=\alpha,\\
    [\hA,[\hA,\hB]]&=[(\alpha^*\ha-\alpha\ha^\dagger),\alpha]=0,\\
    &\vdots
  \end{align*}
  Now, we can express the displacement operator as:
  \begin{align*}
    \tilde{a}&\D^\dagger(\alpha)\ha\D(\alpha)=e^{\alpha^*\ha-\alpha\ha^\dagger}\ha e^{-(\alpha^*\ha-\alpha\ha^\dagger)}=\ha+\alpha,\quad\text{and}\\
    \tilde{a}^\dagger&=\D^\dagger(\alpha)\ha^\dagger\D(\alpha)=\ha^\dagger+\alpha^*.
  \end{align*}
  Application of both in $\ket{0}$ (just to test them) yields:
  \begin{align*}
    \tilde{a}\ket{0}=(\ha+\alpha)\ket{0}=\alpha\ket{0},\quad\text{and}\quad\tilde{a}^\dagger\ket{0}=(\ha^\dagger+\alpha^*)\ket{0}=\ket{1}+\alpha^*\ket{0}.
  \end{align*}
  \item We use algabra of operators and the fact that the coherent state $\ket{\alpha}$ is got from $\ket{0}$ by appliying a displacement operator $\D(\alpha)$:
  \begin{align*}
    \braket{\hat{X}_1}&=\frac{1}{2}\braket{\alpha|(\ha+\ha^\dagger)|\alpha}\\
    &=\frac{1}{2}\braket{0|\D^\dagger(\alpha)(\ha+\ha^\dagger)\D(\alpha)|0}\\
    &=\frac{1}{2}\left[\braket{0|\D^\dagger(\alpha)\ha\D(\alpha)|0}+\braket{0|\D^\dagger(\alpha)\ha^\dagger\D(\alpha)|0}\right]\\
    &=\frac{1}{2}\left[\braket{0|\tilde{a}|0}+\braket{0|\tilde{a}^\dagger|0}\right]\\
    &=\frac{1}{2}\left[\braket{0|(\ha+\alpha)|0}+\braket{0|(\ha^\dagger+\alpha^*)|0}\right]\\
    &=\frac{1}{2}\left[\braket{0|\ha|0}+\alpha+\braket{0|\ha^\dagger|0}+\alpha^*\right]\\
    &=\frac{1}{2}\left[\alpha+\alpha^*\right]\\
    \braket{\hat{X}_1}&=\re{\alpha}.
  \end{align*}
  For $\hat{X}_2$, and using the developmente from above:
  \begin{align*}
    \braket{\hat{X}_2}&=\frac{1}{2i}\left[\braket{0|(\ha+\alpha)|0}-\braket{0|(\ha^\dagger+\alpha^*)|0}\right]\\
    &=\frac{1}{2i}\left[\braket{0|\ha|0}+\alpha-\braket{0|\ha^\dagger|0}-\alpha^*\right]\\
    &=\frac{1}{2i}\left[\alpha-\alpha^*\right]\\
    \braket{\hat{X}_2}&=\im{\alpha}.
  \end{align*}
  For the uncertainties, we now need to get the mean value of $\hat{X}_1^2$ and $\hat{X}_2^2$, with the analogous process:
  \begin{align*}
    \braket{\hat{X}_1^2}=&\frac{1}{4}\braket{0|\D^\dagger(\alpha)[\ha+\ha^\dagger]^2\D(\alpha)|0}\\
    =&\frac{1}{4}\bra{0}\left[\D^\dagger(\alpha)[\ha^2+\ha\ha^\dagger+\ha^\dagger\ha+\ha^{\dagger2}]\D(\alpha)\right]\ket{0}\\
    =&\frac{1}{4}\bra{0}\left[\D^\dagger\ha^2\D(\alpha)+2\D^\dagger(\alpha)\ha^\dagger\ha\D(\alpha)+1+\D^\dagger(\alpha)\ha^{\dagger2}\D(\alpha)\right]\ket{0}\\
    =&\frac{1}{4}\bra{0}\left[(\D^\dagger\ha\D)(\D^\dagger\ha\D)+2(\D^\dagger\ha^\dagger\D)(\D^\dagger\ha\D)+1+(\D^\dagger\ha^\dagger\D)(\D^\dagger\ha^\dagger\D)\right]\ket{0}\\
    =&\frac{1}{4}\bra{0}\left[(\ha+\alpha)^2+2(\ha^\dagger+\alpha^*)(\ha+\alpha)+1+(\ha^\dagger+\alpha^*)^2\right]\ket{0}\\
    =&\frac{1}{4}[\ha^2+2\alpha\ha+\alpha^2+2\ha^\dagger\ha+2\ha^\dagger\alpha+2\alpha^*\ha+2|\alpha|^2+1+\ha^{\dagger2}+2\alpha^*\ha^\dagger+\alpha^{*2}]\\
    \braket{\hat{X}_1^2}=&\frac{1}{4}\left[(\alpha+\alpha^*)^2+1\right].
  \end{align*}
  Then,
  \begin{align*}
    \Delta\hat{X}_1&=\sqrt{\braket{\hat{X}^2_1}-\braket{\hat{X}_1}^2}\\
    &=\sqrt{\frac{1}{4}\left[(\alpha+\alpha^*)^2+1\right]-\frac{1}{4}(\alpha+\alpha^*)^2}\\
    \Delta\hat{X}_1&=\frac{1}{2}.
  \end{align*}
  The same procedure is done for the uncertainty of $\hat{X}_2$:
  \begin{align*}
    \braket{\hat{X}_2}&=\frac{1}{4}\bra{0}\left[(\D^\dagger\ha\D)(\D^\dagger\ha\D)-2(\D^\dagger\ha^\dagger\D)(\D^\dagger\ha\D)-1+(\D^\dagger\ha^\dagger\D)(\D^\dagger\ha^\dagger\D)\right]\ket{0}\\
    &=-\frac{1}{4}\bra{0}\left[(\ha+\alpha)^2-2(\ha^\dagger+\alpha^*)(\ha+\alpha)-1+(\ha^\dagger+\alpha^*)^2\right]\ket{0}\\
    &=-\frac{1}{4}[\ha^2+2\alpha\ha+\alpha^2-2\ha^\dagger\ha-2\ha^\dagger\alpha-2\alpha^*\ha-2|\alpha|^2-1+\ha^{\dagger2}+2\alpha^*\ha^\dagger+\alpha^{*2}]\\
    \braket{\hat{X}_2^2}&=\frac{1}{4}\left[1-(\alpha-\alpha^*)^2\right].
  \end{align*}
  Then,
  \begin{align*}
    \Delta\hat{X}_2&=\sqrt{\braket{\hat{X}^2_2}-\braket{\hat{X}_2}^2}\\
    &=\sqrt{\frac{1}{4}[1-(\alpha-\alpha^*)^2]+\frac{1}{4}(\alpha-\alpha^*)^2}\\
    \Delta\hat{X}_2&=\frac{1}{2}.
  \end{align*}
\end{enumerate}



%\nocite{*}
%\bibliographystyle{plain}   % or unsrt, alpha, apalike, etc.
%\bibliography{refs}

\end{document}
