\documentclass[letterpaper,11pt,twoside]{article}
\usepackage{graphicx} % Required for inserting images
\usepackage[table,xcdraw,dvipsnames]{xcolor}
\usepackage{amsmath,amsfonts,amssymb,amsthm}
\usepackage{listings}
\usepackage{lipsum}
\usepackage{hyperref}
\usepackage{enumitem}

\usepackage{tikz}
\usepackage[siunitx, RPvoltages]{circuitikz}
\usetikzlibrary{3d}
\usepackage{comment}
\usepackage{caption,subcaption}
\usepackage{pgfplots}
\pgfplotsset{compat=newest} % or a newer version if available
\usepgfplotslibrary{groupplots}
\usetikzlibrary{pgfplots.groupplots}
\usetikzlibrary{shapes.geometric, arrows}
\tikzstyle{arrow} = [->,>=stealth,shorten >=2pt]
\newcommand{\re}[1]{\text{Re}\left(#1\right)}
\newcommand{\im}[1]{\text{Im}\left(#1\right)}
\newcommand{\bA}{\bm{A}}
\newcommand{\bB}{\bm{B}}
\newcommand{\bC}{\bm{C}}
\newcommand{\bE}{\bm{E}}
\newcommand{\bH}{\bm{H}}
\newcommand{\bD}{\bm{D}}
\newcommand{\bO}{\bm{0}}
\newcommand{\br}{\bm{r}}
\newcommand{\bR}{\bm{R}}
\newcommand{\hA}{\ha}
\newcommand{\hB}{\hat{B}}
\newcommand{\hC}{\hat{C}}
\newcommand{\ha}{\ha}
\newcommand{\hn}{\bm{\hat{n}}}
\newcommand{\hr}{\bm{\hat{r}}}
\newcommand{\hx}{\bm{\hat{x}}}
\newcommand{\hy}{\bm{\hat{y}}}
\newcommand{\hz}{\bm{\hat{z}}}
\newcommand{\laplace}[1]{\mathscr{L}\left[#1\right]}
\newcommand{\ilaplace}[1]{\mathscr{L}^{-1}\left[#1\right]}
\newcommand{\fourier}[1]{\mathscr{F}\left[#1\right]}
\newcommand{\ifourier}[1]{\mathscr{F}^{-1}\left[#1\right]}
\newcommand{\ket}[1]{\left|#1\right\rangle}
\newcommand{\bra}[1]{\left\langle#1\right|}
\newcommand{\braket}[1]{\left\langle#1\right\rangle}
\newcommand{\D}{\mathcal{D}}
\usepackage{cancel}
\usepackage{bm}
\usepackage{fancyhdr}
\usepackage[utf8x]{inputenc}
\usepackage[T1]{fontenc}
\usepackage[margin=0.8in,top=1in,bottom=1in]{geometry}
%%%%%
\begin{filecontents*}{refs.bib}
@book{bornwolf,
  author    = {Born, M. and Wolf, E.},
  title     = {Principles of Optics},
  publisher = {Pergamon Press},
  edition   = {7},
  year      = {1999}
}
@book{hecht,
  author    = {Hecht, E.},
  title     = {Optics},
  publisher = {Addison-Wesley},
  edition   = {5},
  year      = {2016}
}
\end{filecontents*}
%
\newcommand{\institution}{University of Arizona}
\newcommand{\autor}{Nicolás Hernández Alegría}
\newcommand{\course}{OPTI 544 Quantum Optics}
\newcommand{\assignment}{Assignment 1}
%
\title{\textbf{\assignment}\\\course\\{\Large\institution}}
\author{\autor}
\date{\today\\Total time: 12 hours}
%
\renewcommand{\sectionmark}[1]{\markright{#1}}
\fancypagestyle{mainstyle}{
    \fancyhf{} % Clear all header and footer fields
    \fancyfoot[C]{\thepage}
    \fancyhead[LE,RO]{\course} % Section name on odd pages
    \fancyhead[LO,RE]{\assignment}
    % Optional: Thin rules
    \renewcommand{\headrulewidth}{0pt} % Header rule
    \renewcommand{\footrulewidth}{0pt} % No footer rule
}
%
\begin{document}

\pagestyle{mainstyle}
\maketitle
%%




%\nocite{*}
%\bibliographystyle{plain}   % or unsrt, alpha, apalike, etc.
%\bibliography{refs}



\section*{Exercise 3}
\begin{enumerate}[itemsep=0pt,topsep=0pt,label=\alph*)]
  \item For the displacement operator, we can use the BCH formula 
  \begin{align*}
    e^{\hA+\hB}=e^{\hA}e^{\hB}e^{-\frac{1}{2}[\hA,\hB]}
  \end{align*}
  to get
  \begin{comment}
    \begin{align*}
      \D(\alpha)=e^{\alpha\ha^\dagger-\alpha^*\ha}=e^{\alpha\ha^\dagger}e^{-\alpha^*\ha}e^{\frac{1}{2}|\alpha|^2[\ha^\dagger,\ha]}=e^{|\alpha|^2/2}e^{\alpha\ha^\dagger}e^{\alpha^*\ha}
    \end{align*}
    and 
    \begin{align*}
      \D^\dagger(\alpha)=e^{|\alpha|^2/2}e^{\alpha\ha^\dagger}e^{\alpha^*\ha}.
    \end{align*}
  \end{comment}
  \begin{align*}
    \tilde{a}\ket{0}&=\D^\dagger(\alpha)\ha\D(\alpha)\ket{0}\\
    &=\D^\dagger(\alpha)\ha\ket{\alpha}\\
    &=\D^\dagger(\alpha)\left[e^{-|\alpha|^2/2}\sum_{n=0}^\infty\frac{\alpha^n}{\sqrt{n!}}\ha\ket{n}\right]\\
    &=\D^\dagger(\alpha)\left[e^{-|\alpha|^2/2}\sum_{n=0}^\infty\frac{\alpha^n}{\sqrt{n!}}\ha\ket{n}\right]\\
    &=\D^\dagger(\alpha)\left[e^{-|\alpha|^2/2}\sum_{n=1}^\infty\frac{\alpha^n}{\sqrt{(n-1)!}}\ket{n-1}\right]\\
    &=\D^\dagger(\alpha)\alpha\left[e^{-|\alpha|^2/2}\sum_{m=0}^\infty\frac{\alpha^m}{\sqrt{m!}}\ket{m}\right]\quad(m=n-1)\\
    &=\alpha\D^\dagger(\alpha)\ket{\alpha}.
  \end{align*}
  Taking adjoint of the displacement action allow us to get the action of its adjoint: 
  \begin{align*}
    \D(\alpha)\ket{0}&=\ket{\alpha}\bigr/\D^\dagger(\alpha)\\
    \D^\dagger(\alpha)\D(\alpha)\ket{0}&=\D^\dagger(\alpha)\ket{\alpha}\\
    \ket{0}&=\D^\dagger(\alpha)\ket{\alpha}.
  \end{align*}
  Then, we have:
  \begin{align*}
    \tilde{a}\ket{0}=\alpha\ket{0}.
  \end{align*}

  For $\tilde{a}^\dagger=\D^\dagger(\alpha)\ha^\dagger\D(\alpha)$, we need to know the action of $\ha^\dagger$:
  \begin{align*}
    \ha^\dagger\ket{\alpha}=e^{-|\alpha|^2/2}\sum_{n=0}^\infty\frac{\alpha^n}{\sqrt{n!}}\ha^\dagger\ket{n}=e^{-|\alpha|^2/2}\sum_{n=0}^\infty\frac{\alpha^n}{\sqrt{n!}}\sqrt{n+1}\ket{n+1}=e^{-|\alpha|^2/2}\sum_{m=1}^\infty\frac{\sqrt{m}\alpha^{m-1}}{\sqrt{(m-1)!}}\ket{m}.
  \end{align*}

  \item sagag
\end{enumerate}
\end{document}
